%%% PREAMBLE %%%

% Strict mode

\RequirePackage[l2tabu, orthodox]{nag} % Warn when using deprecated constructs

% Document class and packages

\documentclass[10pt,a4paper,parskip,fleqn]{scrartcl}
\usepackage[a4paper,vmargin={30mm},hmargin={30mm}]{geometry} % Page margins
\usepackage[ngerman]{babel} % New German hyphenation (multilingual support)
\usepackage[utf8x]{inputenc} % Unicode support
\usepackage{graphicx} % Graphics support
\usepackage{enumitem}
\usepackage{verbatimbox}
\usepackage{xspace}
\usepackage{units}

% Font configuration

\usepackage[sc]{mathpazo} % Use palatino font
\usepackage[T1]{fontenc} % Use correct font encoding
\usepackage[babel=true]{microtype} % Micro-typographic optimizations
\addtokomafont{disposition}{\rmfamily} % Set palatino as heading font

% Section Titles

\usepackage{titlesec}
\titleformat{\section}{\Large\bfseries}{§ \thesection{} -- }{0em}{}

% Numbering

\renewcommand*{\labelenumi}{\thesection.\arabic{enumi}}
\renewcommand*{\labelenumii}{\labelenumi.\arabic{enumii}}

% Easier enumerations

\newcommand{\ol}{\begin{enumerate}[itemsep=-0.2em,topsep=-0.2em]}
\newcommand{\lo}{\end{enumerate}}
\newcommand{\li}{\item}



%%% VARIABLES %%%

\newcommand{\name}{Chaostreff Coredump\xspace}
\newcommand{\iname}{\textit{\name}\xspace}
\newcommand{\parent}{Coredump Rapperswil\xspace}
\newcommand{\iparent}{\textit{\parent}\xspace}
\newcommand{\cccch}{CCC-CH\xspace}
\newcommand{\icccch}{\textit{\cccch}\xspace}



%%% TITLEPAGE %%%

\title{\Huge Vereinsstatuten}
\subject{\name}
\date{Stand: 2016-06-11}



%%% HEADER / FOOTER %%%

\usepackage{fancyhdr} % Fancy headers
\usepackage{lastpage} % Page numbering

% Capture title and author
\makeatletter
\let\Date\@date
\makeatother

% Fancy headers configuration
\pagestyle{fancy}
\fancyhead{} % Clear all header fields
\fancyhead[LO,LE]{\bfseries Vereinsstatuten \name}
\fancyhead[RO,RE]{\bfseries \Date{}}
\fancyfoot{} % Clear all footer fields
\fancyfoot[CO,CE]{Seite \thepage}
\renewcommand{\headrulewidth}{0.3pt}
\renewcommand{\footrulewidth}{0pt}

% Better footer
\cfoot{Seite \thepage\ von \pageref{LastPage}}


%%% MAIN DOCUMENT %%%

\begin{document}

\begin{titlepage}

	\maketitle
	\thispagestyle{empty} % Don't start page numbers on this page

  \begin{center}

		\vspace{1cm}

		\theverbbox

		\vfill

		\large Beschlossen an der Gründungsversammlung vom 2016-06-11.

		\vspace{1.5cm}

		\begin{minipage}[t]{0.49\textwidth}
			\center
			\rule{5cm}{0.2mm}\\
			Danilo Bargen, für den Vorstand
		\end{minipage}
		\begin{minipage}[t]{0.49\textwidth}
			\center
			\rule{5cm}{0.2mm}\\
			Raphael Nestler, für den Vorstand
		\end{minipage}

  \end{center}

\end{titlepage}


\section{Name, Sitz \& Zweck}

\ol
	\li Unter dem Namen \iname besteht ein Verein im Sinne von Art. 60
	ff. ZGB mit Sitz in Rapperswil-Jona.
	\li Der Verein ist parteipolitisch und religiös neutral.
	\li Er bezweckt den die Vertretung der Mitglieder im \icccch.
\lo


\section{Mitgliedschaft}

\ol
  \li Jede natürliche Person, die im \iparent Mitglied ist und sich den
	Zielsetzungen des \icccch verbunden fühlt, kann Mitglied im \iname werden.
	\li Aufnahmegesuche sind an ein Vorstandsmitglied zu richten; über die
	Aufnahme entscheidet der Vorstand.
	\li Mitglieder können per Monatsende aus dem Verein austreten. Dazu ist eine
	schriftliche Mitteilung an den Vorstand notwendig. Bereits geschuldete
	Beiträge werden nicht erstattet.
	\li Die Mitgliedschaft erlischt durch Austritt, Ausschluss oder Tod.
	\li Ein Mitglied kann jederzeit ohne Angabe eines Grundes aus dem Verein
	ausgeschlossen werden. Der Vorstand fällt den Ausschlussentscheid; das
	Mitglied kann den Ausschlussentscheid an die Generalversammlung weiterziehen.
\lo


\section{Organe des Vereins}

\ol
	\li Die Organe des Vereins sind:
		\ol
			\li die Generalversammlung
			\li der Vorstand
		\lo
\lo


\section{Die Generalversammlung}

\ol
	\li Die Generalversammlung ist das höchste Organ des Vereins. Sie beschliesst
	mit einfacher Mehrheit der abgegebenen Stimmen:
		\ol
			\li Budgets und Mitgliederbeiträge inkl. Periodizität
		\lo
	\li Sie beschliesst mit qualifiziertem Mehr von $\nicefrac{2}{3}$ der
	abgegebenen Stimmen:
		\ol
			\li Ausschluss von Mitgliedern
			\li Änderung der Statuten
			\li Auflösung des Vereines
		\lo
	\li Bei Stimmengleichheit entscheidet der Präsident mit Stichentscheid.
	\li Die Generalversammlung des \iname findet automatisch anschliessend an die
	Generalversammlung des \iparent statt.
	\li Die Generalversammlung kann die Traktandenliste anpassen und über neue
	Traktanden beschliessen; ausgenommen den Ausschluss von Mitgliedern.
\lo


\section{Der Vorstand}

\ol
	\li Der Vorstand des Vereins \iparent amtiert gleichzeitig als Vorstand des
	\iname.
	\li Er vertritt den Verein gegen aussen und führt die laufenden Geschäfte,
	insbesondere die Kommunikation mit dem \icccch.
\lo


\section{Finanzen}

\ol
	\li Der Verein finanziert sich aus den Mitgliederbeiträgen sowie allfälligen
	Spenden und Legaten.
	\li Der Mitgliederbeitrag wird jährlich erhoben. Die Höhe des
	Mitgliederbeitrags wird jährlich von der Generalversammlung festgelegt.
	\li Für Personen, die dem Verein vor dem 1. August beitreten ist der volle
	Mitgliederbeitrag für das laufende Jahr zu erstatten. Tritt eine Person aber
	nach dem 1. August bei, so muss für das laufende Jahr kein Mitgliederbeitrag
	entrichtet werden.
	\li Das Vereinsjahr beginnt am 1. Oktober und endet am 30. September.
\lo


\section{Unterschrift, Haftung}

\ol
	\li Der Verein wird verpflichtet durch Einzelunterschrift, ausgenommen das
	Eingehen von Dauerschuldverhältnissen kollektiv zu zweien. Die Aufnahme von
	Krediten, das Leisten von Bürgschaften sowie öffentliche Beurkundungen
	bedürfen zusätzlich der Bewilligung durch die Generalversammlung.
	\li Für die Schulden des Vereins haftet nur das Vereinsvermögen. Eine
	persönliche Haftung der Mitglieder ist ausgeschlossen.
\lo


\section{Auflösung des Vereins}

\ol
	\li Bei einer Auflösung des Vereins fällt das Vereinsvermögen an den \iparent.
\lo


\section{Übergangs- und Vollzugsbestimmungen}

\ol
	\li Diese Statuten sind an der Gründungsversammlung vom 11. Juni 2016
	angenommen worden und sind mit diesem Datum in Kraft getreten.
	\li Bei Unklarheiten über die Auslegung dieser Statuten entscheidet der
	Vorstand abschliessend.
\lo


\end{document}
