%%% PREAMBLE %%%

% Strict mode

\RequirePackage[l2tabu, orthodox]{nag} % Warn when using deprecated constructs

% Document class and packages

\documentclass[10pt,a4paper,parskip,fleqn]{scrartcl}
\usepackage[a4paper,vmargin={30mm},hmargin={30mm}]{geometry} % Page margins
\usepackage[ngerman]{babel} % New German hyphenation (multilingual support)
\usepackage[utf8x]{inputenc} % Unicode support
\usepackage{graphicx} % Graphics support
\usepackage{enumitem}
\usepackage{verbatimbox}

% Font configuration

\usepackage[sc]{mathpazo} % Use palatino font
\usepackage[T1]{fontenc} % Use correct font encoding
\usepackage[babel=true]{microtype} % Micro-typographic optimizations
\addtokomafont{disposition}{\rmfamily} % Set palatino as heading font

% Section Titles

\usepackage{titlesec}
\titleformat{\section}{\Large\bfseries}{§ \thesection{} -- }{0em}{}

% Numbering

\renewcommand*{\labelenumi}{\thesection.\arabic{enumi}}
\renewcommand*{\labelenumii}{\labelenumi.\arabic{enumii}}

% Easier enumerations

\newcommand{\ol}{\begin{enumerate}[itemsep=-0.2em,topsep=-0.2em]}
\newcommand{\lo}{\end{enumerate}}
\newcommand{\li}{\item}



%%% TITLEPAGE %%%

\title{\Huge Vereinsstatuten}
\subject{coredump}
\date{Stand: 26. August 2013}



%%% HEADER / FOOTER %%%

\usepackage{fancyhdr} % Fancy headers
\usepackage{lastpage} % Page numbering

% Capture title and author
\makeatletter
\let\Date\@date
\makeatother

% Fancy headers configuration
\pagestyle{fancy}
\fancyhead{} % Clear all header fields
\fancyhead[LO,LE]{\bfseries Vereinsstatuten coredump}
\fancyhead[RO,RE]{\bfseries \Date{}}
\fancyfoot{} % Clear all footer fields
\fancyfoot[CO,CE]{Seite \thepage}
\renewcommand{\headrulewidth}{0.3pt}
\renewcommand{\footrulewidth}{0pt}

% Better footer
\cfoot{Seite \thepage\ von \pageref{LastPage}}


%%% MAIN DOCUMENT %%%

\begin{document}

% Logo in ASCII-art
\begin{verbbox}
  ___ ___  _ __ ___  __| |_   _ _ __ ___  _ __  
 / __/ _ \| '__/ _ \/ _` | | | | '_ ` _ \| '_ \ 
| (_| (_) | | |  __/ (_| | |_| | | | | | | |_) |
 \___\___/|_|  \___|\__,_|\__,_|_| |_| |_| .__/ 
                                         |_|    
\end{verbbox}

\begin{titlepage}

	\maketitle
	\thispagestyle{empty} % Don't start page numbers on this page

  \begin{center}

		\vspace{1cm}

		\theverbbox

		\vfill

		\large Beschlossen an der Gründungsversammlung vom 26. August 2013.

		\vspace{1.5cm}

		\begin{minipage}[t]{0.49\textwidth}
			\center
			\rule{5cm}{0.2mm}\\
			Danilo Bargen, Präsident
		\end{minipage}
		\begin{minipage}[t]{0.49\textwidth}
			\center
			\rule{5cm}{0.2mm}\\
			Josua Schmid, Kassier
		\end{minipage}

  \end{center}

\end{titlepage}


\section{Name und Sitz}

\ol
	\li Unter dem Namen \textit{coredump} besteht ein Verein im Sinne von Art. 60
	ff. ZGB.
	\li Der Sitz des Vereins ist Rapperswil-Jona. Für alle Rechtsstreitigkeiten
	gilt schweizerisches Recht.
\lo


\section{Vereinszweck}

\ol
	\li Der Verein verfolgt folgende Ziele:
    \ol
      \li Bereitstellung der Infrastruktur zur Arbeit an nicht-kommerziellen
			technischen Projekten
      \li Förderung des Informationsaustausches an regelmässigen Treffen,
			speziell im Bereich der Technologie
    \lo
		\li Der Verein handelt im Sinne der
		Hackerethik\footnote{Siehe \texttt{http://www.ccc.de/de/hackerethik}}. Er
		ist parteipolitisch und konfessionell neutral.
\lo


\section{Mitgliedschaft}

\ol
  \li Aktivmitglied mit Stimmberechtigung kann jede natürliche Person werden,
	die sich mit dem Vereinszweck identifizieren kann.
	\li Passivmitglied ohne Stimmberechtigung kann jede natürliche und juristische
	Person werden.
	\li Aufnahmegesuche sind an ein Vorstandsmitglied zur richten; über die
	Aufnahme entscheidet der Vorstand.
	\li Die Mitgliedschaft erlischt
		\ol
			\li bei natürlichen Personen durch Austritt, Ausschluss oder Tod
			\li bei juristischen Personen durch Austritt, Ausschluss oder Auflösung
		\lo
\lo


\section{Austritt und Ausschluss}

\ol
	\li Ein Vereinsaustritt ist per Ende des Vereinsjahrs möglich. Das
	Austrittsschreiben muss schriftlich an den Vorstand gerichtet werden.
	\li Ein Mitglied kann jederzeit ohne Grundangabe aus dem Verein ausgeschlossen
	werden. Der Vorstand fällt den Ausschlussentscheid; das Mitglied kann den
	Ausschlussentscheid an die Generalversammlung weiterziehen.
\lo


\section{Organe des Vereins}

\ol
	\li Die Organe des Vereins sind:
		\ol
			\li die Generalversammlung
			\li der Vorstand
		\lo
\lo


\section{Die Generalversammlung}

\ol
	\li Das oberste Organ des Vereins ist die Generalversammlung. Eine ordentliche
	Generalversammlung findet jährlich statt.
	\li Zur Generalversammlung werden die Mitglieder drei Wochen zum voraus
	schriftlich eingeladen, unter Beilage der Traktandenliste.
	\li Die Generalversammlung hat die folgenden unentziehbaren Aufgaben:
		\ol
			\li Wahl bzw. Abwahl des Vorstandes
			\li Festsetzung und Änderung der Statuten
			\li Abnahme der Jahresrechung
			\li Beschluss über das Jahresbudget
			\li Festsetzung der Mitgliederbeiträge
			\li Behandlung der Ausschlussrekurse
		\lo
	\li An der Generalversammlung besitzt jedes Aktivmitglied eine Stimme; die
	Beschlussfassung erfolgt mit einfachem Mehr.
	\li Bei Stimmengleichheit entscheidet der Präsident mit Stichentscheid.
	\li Passivmitglieder werden zur Generalversammlung eingeladen, besitzen jedoch
	kein Stimmrecht.
\lo


\section{Der Vorstand}

\ol
	\li Der Vorstand besteht aus mindestens zwei Personen, nämlich dem Präsidenten
	und dem Kassier.
	\li Der Vorstand vertritt den Verein nach aussen und führt die laufenden
	Geschäfte.
\lo


\section{Finanzen}

\ol
	\li Der Mitgliederbeitrag wird jährlich erhoben. Die Höhe des
	Mitgliederbeitrags wird jährlich von der Vereinsversammlung festgelegt.
	\li Tritt ein Mitglied während des Vereinsjahrs bei, wird der
	Mitgliederbeitrag pro rata für die verbleibenden Monate erhoben.
	\li Das Vereinsjahr entspricht dem Kalenderjahr.
\lo


\section{Unterschrift, Haftung}

\ol
	\li Der Verein wird verpflichtet durch die Kollektivunterschrift des
	Präsidenten zusammen mit einem weiteren Mitglied des Vorstandes.
	\li Für die Schulden des Vereins haftet nur das Vereinsvermögen. Eine
	persönliche Haftung der Mitglieder ist ausgeschlossen.
\lo


\section{Statutenänderungen}

\ol
	\li Die vorliegenden Statuten können abgeändert werden, wenn eine Mehrheit der
	anwesenden stimmberechtigten Mitglieder dem Änderungsvorschlag zustimmt.
\lo


\section{Auflösung des Vereins}

\ol
	\li Die Auflösung des Vereins kann an einer Vereinsversammlung mit absoluter
	Mehrheit der anwesenden stimmberechtigten Mitglieder beschlossen werden.
	\li Bei einer Auflösung des Vereins fällt das Vereinsvermögen an eine
	Institution, welche den gleichen oder einen ähnlichen Zweck verfolgt.
\lo


\section{Übergangs- und Vollzugsbestimmungen}

\ol
	\li Diese Statuten sind an der Gründungsversammlung vom 26. August 2013
	angenommen worden und sind mit diesem Datum in Kraft getreten.
	\li Bei Unklarheiten über die Auslegung dieser Statuten entscheidet der
	Vorstand abschliessend.
\lo


\end{document}
